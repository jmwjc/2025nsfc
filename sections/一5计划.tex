\subsection{年度研究计划}
本项目围绕拟定的研究目标,通过理论推导和数值验证的方法研究任意节点分布的时空混合离散伽辽金无网格法。
根据各部分研究内容的逻辑关系,制定了表\ref{tb:plan}所示的项目的年度计划表。
其中,成果整理、论文发表及年度报告贯穿执行期。
项目参与人执行期内计划参加学术会议并作项目相关报告不少于每年2人次。

\begin{table}[!h]
\caption{年度研究计划表}\label{tb:plan}
\centering
\begin{ganttchart}[
    % inline,
    % bar inline label anchor = north,
    include title in canvas=false,
    x unit = 8pt,
    y unit title = 14pt,
    hgrid={*1{draw=black!5, line width=.75pt}},
    vgrid={*1{draw=black!5, line width=.75pt}},
    canvas/.append style={fill=none,draw=black!20,line width=.75pt},
    title/.style={draw=black!5, fill=none},
    title height=1,
    bar height = 0.3,
    group height=.1,
    % title label node/.append style={below=0pt},
    newline shortcut=true,
    % bar label node/.append style={align=left, anchor=north},
    bar/.append style={fill=blue!50, rounded corners=4pt, draw=none},
    group label node/.append style={align=left, left=-.5cm, text width=17em},
    bar label node/.append style={left=0cm,align=left,text width=15em},
]{1}{32}
    \gantttitle{2026}{8}
    \gantttitle{2027}{8}
    \gantttitle{2028}{8}
    \gantttitle{2029}{8} \\
    \gantttitle{上半年}{4}
    \gantttitle{下半年}{4}
    \gantttitle{上半年}{4}
    \gantttitle{下半年}{4}
    \gantttitle{上半年}{4}
    \gantttitle{下半年}{4}
    \gantttitle{上半年}{4}
    \gantttitle{下半年}{4} \\
    \ganttgroup{\bfseries 研究内容(1)}{1}{8} \\
    \ganttbar{查阅相关文献,测试不同虚位移边界条件时空混合离散伽辽金法性能}{1}{4}\\
    \ganttbar[name=scheme1]{研究拉格朗日乘子型时域末端虚位移本质边界条件施加方案}{3}{8}\\
    \ganttgroup{\bfseries 研究内容(2)}{9}{20} \\
    \ganttbar[name=error]{推导稳定性分析局部截断误差估计}{9}{20} \\
    \ganttbar[name=scheme2]{构建时空混合离散再生核近似方案}{13}{16} \\
    \ganttgroup{\bfseries 研究内容(3)}{17}{26} \\
    \ganttbar[name=scheme3]{推导时空混合离散再生光滑梯度}{17}{24} \\
    \ganttbar[name=scheme4]{优化四维积分域再生光滑梯度积分方案}{21}{26} \\
    \ganttgroup{\bfseries 研究内容(4)}{21}{28} \\
    \ganttbar[name=method]{构建时空混合离散伽辽金无网格法}{21}{24} \\
    \ganttbar{引入自适应节点分布算法和并行计算优化程序}{23}{28} \\
    \ganttbar{成果整理、论文发表、软件著作权申请、年度报告及项目结题报告}{5}{32}
\end{ganttchart}
\end{table}

\subsection{预期研究结果}
本项目预期取得如下研究结果:

\begin{enumerate}[label=(\theenumi),left=24pt]
    \item 时空混合离散伽辽金法时域末端本质边界条件施加方案;
    \item 免数值色散问题的时空混合离散再生核无网格近似方案;
    \item 绝对混合离散高效伽辽金无网格分析方法;
    \item 在计算力学领域重要期刊发表论文5--9篇,申请软件著作权1项;
    \item 培养研究生不少于4名。
\end{enumerate}
